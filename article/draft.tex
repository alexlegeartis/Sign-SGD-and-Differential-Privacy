\documentclass{article}

% More detailed margin control
\usepackage[left=1in,right=1in,top=1in,bottom=1in]{geometry}

% Essential packages
\usepackage{amsmath}
\usepackage[utf8]{inputenc} % allow utf-8 input
\usepackage[T1]{fontenc}    % use 8-bit T1 fonts
\usepackage{lmodern}
\usepackage{hyperref}       % hyperlinks
\usepackage{url}            % simple URL typesetting
\usepackage{booktabs}       % professional-quality tables
\usepackage{amsfonts}       % blackboard math symbols
\usepackage{nicefrac}       % compact symbols for 1/2, etc.
\usepackage{microtype}      % microtypography
\usepackage{xcolor}         % colors
\usepackage{comment}
\usepackage{enumitem}
\usepackage{subcaption}
\usepackage{graphicx}
\usepackage{amsthm}

% Compatibility for biblatex
\usepackage{csquotes}

% Load biblatex before cleveref
\usepackage{biblatex}
\bibliography{draft_lib.bib}

\usepackage{cleveref}

% Theorem environments
\newtheorem{assumption}{Assumption}
\newtheorem{lemma}{Lemma}
\newtheorem{proposition}{Proposition}
\newtheorem{theorem}{Theorem}
\newtheorem{corollary}{Corollary}
\newtheorem{definition}{Definition}
\newtheorem{remark}{Remark}

% Comments for co-authors (optional)
\newcommand{\coauthorcomment}[2]{{\color{#1} \textbf{#2}}}

% Title and author information
\title{Sign SGD with Heavy-Tailed Noise and Differential Privacy}

\author{
  Alexey Kravatskiy\\
  \texttt{kravtskii.aiu@phystech.edu}
  \and
  Anton Plusnin\\
  \texttt{plusnin.aa@phystech.edu}
  \and
  Savelii Chezhegov\\
  \texttt{chezhegov.sa@phystech.edu}
}

\date{\today}

\begin{document}

\maketitle

\begin{abstract}
Federated learning faces the dual challenge of ensuring fast communication and maintaining data privacy. This paper addresses these challenges by proposing a novel modification of the Sign-SGD algorithm, designed to be both communication-efficient and differentially private. Our approach is particularly suited for real-world data characterized by noise distributions with infinite variance and bounded k-th moments, where \( k \) ranges from 1 to 2. We demonstrate that our modified Sign-SGD algorithm achieves differential privacy and converges with high probability, even when the noise is heavy-tailed, asymmetric, or non-unimodal. Additionally, we show that the algorithm's speed is comparable to existing methods. The effectiveness of our approach is validated through improved training quality and speed in Large Language Models.
\end{abstract}

\paragraph{Keywords:} Sign SGD, differential privacy, high-probability convergence, federated learning, heavy-tailed noise.

\paragraph{Highlights:}
\begin{enumerate}
\item Sign Stochastic Gradient Descent can be used to train LLMs on real data.
\item Our modification of Sign Stochastic Gradient Descent keeps user data private.
\item Our modification of Sign Stochastic Gradient Descent does not require tuning.
\end{enumerate}

\section{Introduction}
In the recent work \parencite{Kornilov2025}, the authors have shown that Sign-SGD with heavy-tailed noise achieves optimal convergence rates under certain assumptions. Additionally, \parencite{Jin2020} demonstrated that Sign-SGD can be made differentially private while maintaining good convergence properties.

\section{Theory}
Present your theoretical framework, definitions, lemmas, and proofs.

\section{Experiments}
Describe your experimental setup, methodology, and results.

\section{Conclusion}
Summarize your findings and discuss future work.

\section*{Acknowledgments}
Optional acknowledgments section.

\appendix
\section{Additional Proofs and Results}
Include detailed proofs and supplementary materials here.

\printbibliography

\end{document}