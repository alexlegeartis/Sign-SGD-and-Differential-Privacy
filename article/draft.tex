\documentclass{article}

% Essential packages
\usepackage{amsmath}
\usepackage[utf8]{inputenc} % allow utf-8 input
\usepackage[T1]{fontenc}    % use 8-bit T1 fonts
\usepackage{lmodern}
\usepackage{hyperref}       % hyperlinks
\usepackage{url}            % simple URL typesetting
\usepackage{booktabs}       % professional-quality tables
\usepackage{amsfonts}       % blackboard math symbols
\usepackage{nicefrac}       % compact symbols for 1/2, etc.
\usepackage{microtype}      % microtypography
\usepackage{xcolor}         % colors
\usepackage{comment}
\usepackage{enumitem}
\usepackage{subcaption}
\usepackage{graphicx}
\usepackage{amsthm}

% Compatibility for biblatex
\usepackage{csquotes}

% Load biblatex before cleveref
\usepackage{biblatex}
\bibliography{draft_lib.bib}

\usepackage{cleveref}

% Theorem environments
\newtheorem{assumption}{Assumption}
\newtheorem{lemma}{Lemma}
\newtheorem{proposition}{Proposition}
\newtheorem{theorem}{Theorem}
\newtheorem{corollary}{Corollary}
\newtheorem{definition}{Definition}
\newtheorem{remark}{Remark}

% Comments for co-authors (optional)
\newcommand{\coauthorcomment}[2]{{\color{#1} \textbf{#2}}}

% Title and author information
\title{Sign SGD with Heavy-Tailed Noise and Differential Privacy}

\author{
  Alexey Kravatskiy\\
  \texttt{kravtskii.aiu@phystech.edu}
  \and
  Anton Plusnin\\
  \texttt{plusnin.aa@phystech.edu}
  \and
  Savelii Chezhegov\\
  \texttt{chezhegov.sa@phystech.edu}
}

\date{\today}

\begin{document}

\maketitle

\begin{abstract}
For now, see a separate file for it.
\end{abstract}

\section{Introduction}
In the recent work \parencite{Kornilov2025}, the authors have shown that Sign-SGD with heavy-tailed noise achieves optimal convergence rates under certain assumptions. Additionally, \parencite{Jin2020} demonstrated that Sign-SGD can be made differentially private while maintaining good convergence properties.

\section{Theory}
Present your theoretical framework, definitions, lemmas, and proofs.

\section{Experiments}
Describe your experimental setup, methodology, and results.

\section{Conclusion}
Summarize your findings and discuss future work.

\section*{Acknowledgments}
Optional acknowledgments section.

\appendix
\section{Additional Proofs and Results}
Include detailed proofs and supplementary materials here.

\printbibliography

\end{document}